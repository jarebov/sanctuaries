\documentclass[xcolor=pdftex,dvipsnames,table]{beamer}
%\documentclass[xcolor=pdftex,dvipsnames,table,handout]{beamer}
% EEHPT: You can change the theme color from Green to other colors.
\usecolortheme[named=Blue]{structure}
\useinnertheme{circles}
\setbeamertemplate{enumerate items}[default]
\setbeamertemplate{section in toc}[default]
%\usetheme{Default}
%\usepackage{times}
%\setbeamerfont{title}{shape=\itshape,family=\rmfamily}
\setbeamercovered{invisible}
\usepackage{wrapfig}
\usepackage[absolute,overlay]{textpos}
\usepackage{array}
\usepackage{threeparttablex}
\newcolumntype{C}[1]{>{\centering\let\newline\\\arraybackslash\hspace{0pt}}m{#1}}
\newcolumntype{L}[1]{>{\raggedright\let\newline\\\arraybackslash\hspace{0pt}}m{#1}}
%% THE FOLLOWING TO HAVE TIMES NEW ROMAN
%\setbeamertemplate{frametitle}{
%\begin{centering} 
%\rmfamily\scshape \insertframetitle\par\vskip-6pt%\hrulefill
%\end{centering}} 
%\usefonttheme{serif}

\setbeamertemplate{frametitle}{
\begin{centering} 
\insertframetitle\par%\hrulefill
\end{centering}}


\usepackage {tikz}
\usepackage{textcomp}
\usetikzlibrary{positioning}
%\usepackage {xcolor}
\usetikzlibrary{shapes,arrows}

\usepackage{multirow}
%\usepackage[T1]{fontenc}
%\usepackage{txfonts}
\newcommand\Fontv{\fontsize{5}{5.5}\selectfont}
\newcommand\Fontvi{\fontsize{6}{7.4}\selectfont}
\newcommand\Fontvii{\fontsize{7}{8.4}\selectfont}
\newcommand\Fontviii{\fontsize{8}{9.6}\selectfont}
\newcommand*\itemgray{%
  \item[\color{gray}\scalebox{0.9}{\textbullet}]}
  
  
  
  %%% THE LINE BELOW MUST BE CHANGED WITH FOLDER IN USE!!
\newcommand{\tablesfolder}{/Users/bbiasi/Dropbox/Research/sanctuaries/sanctuaries_git/tex/tables}  


  %%% THE LINE BELOW MUST BE CHANGED WITH FOLDER IN USE!!
\graphicspath{{/Users/bbiasi/Dropbox/Research/sanctuaries/out/}}

%\usepackage{lmodern}
%\usepackage[beamer,customcolors]{hf-tikz}
%\usepackage{enumitem}
%\usepackage{xcolor}
%\newif\ifgooditem
%\gooditemtrue
%\newcommand\gooditem{\gooditemtrue\item}
%\newcommand\baditem{\gooditemfalse\item}

%\tikzset{hl/.style={
%    set fill color=white,
%    set border color=red!80!black,
%  },
%}  
%\setbeamertemplate{footline}[default]
\setbeamercolor{button}{bg=white,fg=Blue}
\setbeamertemplate{button}[circle]

\usepackage{tikz}
\usepackage{bbding}
%\usepackage{pifont}
\usepackage{wasysym}
\usepackage{amssymb}
\usepackage{tikz}
\usepackage{booktabs}
\usepackage{multirow}
\setbeamertemplate
 {footline}{\nonumber}
  
\AtBeginSection[]
{
   \begin{frame}
        \frametitle{Outline}
        \tableofcontents[currentsection, 
    hideothersubsections]
   \end{frame}
}

\AtBeginSubsection[]
{
   \begin{frame}
        \frametitle{Outline}
        \tableofcontents[currentsection,currentsubsection]
   \end{frame}
}

\setbeamercolor{section in toc}{fg=Blue}
\setbeamercolor{subsection in toc}{fg=Blue}
\setbeamertemplate{section in toc}[sections numbered]
\setbeamerfont{subsection in toc}{size=\small}




\author[Short Name (U ABC)]{%
  \texorpdfstring{%
    \begin{columns}
      \column{.3333\linewidth}
      \centering
      Jaime Arellano-Bover \\ Stanford
      \column{.3333\linewidth}
      \centering
      Barbara Biasi \\ Princeton
    \end{columns}
 }
 {Jaime Arellano-Bover, Barbara Biasi}
}


\beamertemplatenavigationsymbolsempty
\begin{document}

\title[Sanctuary cities and crime]{Sanctuary Cities and Crime}

\date{December 2017}

\maketitle

\begin{frame}{Sanctuary Cities}
\begin{itemize}
\item Policies and practices of local authorities and law enforcement agencies (LEA) to restrict cooperation w/ federal authorities (FA) to enforce immigration law\vspace{0.10cm}
\begin{itemize}
\item Inquiries about immigration status\vspace{0.10cm}
\item Detainment of arrested individuals past release date to favor removal by FA\vspace{0.10cm}
\item Training of local officials to enforce immigration law
\end{itemize}\vspace{0.20cm}
\item Date back to 1980: religious buildings as safe heavens for refugees from Central America\vspace{0.20cm}
\item Jan 2017: EO 13768, federal government to defund sanctuary jurisdictions
\end{itemize}
\end{frame}

\begin{frame}{Debate over Sanctuaries}
Legal Framework\vspace{0.20cm}
\begin{itemize}
\item 8 US code 1373, no restrictions on LEA-FA communication and info exchange
\item 8 US code 1324, prohibits shielding illegal aliens from detection by FA
\item  LEA should not be forced to do FA's job
\end{itemize}\vspace{0.20cm}
Debate
\begin{itemize}
\item Against: They harm public safety,  release of criminals who should be deported
\item In favor: Reinforce trust between LEA and community, improves policing activities
\end{itemize}
\end{frame}

\begin{frame}{ Do sanctuaries affect crime?}
\begin{itemize}
\item In general: Does cooperation between LEA and FA on enforcement of immigration law affects crime?\vspace{0.20cm}
\item Today: Focus on refusals of ICE detainers\vspace{0.20cm}
\begin{itemize}
\item Refusal to hold ind's past release date to favor ICE's intervention\vspace{0.10cm}
\item Introduced in 100 counties between 2008 and 2014\vspace{0.10cm}
\item Compare crime no-detainer counties vs. bordering counties\vspace{0.10cm}
\item (future) Synthetic controls, matching
\end{itemize}\vspace{0.20cm}
\item Future: look at 287(g) agreements - training of local officials
\end{itemize}
\end{frame}


\begin{frame}{Literature}
\begin{itemize}
\item \textbf{\textcolor{Blue}{Immigration and crime}}\vspace{0.10cm}
\begin{itemize}
\item Correlation: Butcher and Piel (2007), Reid et al. (2005), Graif and Sampson (2009)\vspace{0.10cm}
\item Causal: Bell Fasani and Machin (2013), Chalfin (2014)
\end{itemize}\vspace{0.20cm}
\item \textbf{\textcolor{Blue}{Immigration \textit{status} and crime}}: Freedman Owens and Bohn (2014), Baker (2014), Pinotti (2017)\vspace{0.20cm}
\item \textbf{\textcolor{Blue}{Deportation and crime}}: Miles and Cox (2014, 2015) \vspace{0.10cm}
\begin{itemize}
\item Secure Communities (more on it later) \vspace{0.10cm}
\item Zero effect on crime (2014) and police effectiveness (2015)
\end{itemize}\vspace{0.20cm}
\item \textbf{\textcolor{Blue}{Sanctuary cities and crime}}: Gonzales Collingwood and El-Khatib (2017) \vspace{0.10cm}
\begin{itemize}
\item Cross-sectional comparison (2012), city-level, no correlation
\end{itemize}
\end{itemize}
\end{frame}


\section{Institutional background}
\begin{frame}{Historical context}
\begin{itemize}
\item State and local police primary enforcers of criminal laws in US\vspace{0.20cm}
\item Historically, immigration is fed competence\vspace{0.10cm}
\begin{itemize}
\item Dept of Homeland Security - Immigration \& Custom Enforcement \vspace{0.10cm}
\item Border security, immigration law
\end{itemize}\vspace{0.20cm}
\item 1996: Immigration and Nationality Act, Section 287(g)\vspace{0.10cm}
\begin{itemize}
\item FA train LEA officers to enforce immigration law\vspace{0.10cm}
\item Widely used until 2008
\end{itemize}\vspace{0.20cm}
\item 2008: Secure Communities\vspace{0.10cm}
\begin{itemize}
\item Limit LEA discretion\vspace{0.10cm}
\item LEAs required to share info on arrested non-citizens with ICE \vspace{0.10cm}
\item Removals peak at 400,000/year in 2014
\end{itemize}
% Note: Miles&Cox (2014) use SecureCommunities program to study direct effect on crime. The 2015 article checks whether police officers find crimes at a higher rate, to test for cooperation. Both papers have a zero result
\end{itemize}
\end{frame}


\begin{frame}{Secure Communities}
\begin{itemize}
\item Established in 2008 by Obama administration \vspace{0.20cm}
\item Idea: reduce LEAs discretion, rely on local systems to remove noncitizen criminals \vspace{0.20cm}
\item Disclosure of info on arrested ind's from LEAs to ICE
\begin{itemize} \vspace{0.10cm}
\item Before: arrest $\rightarrow$ fingerprints to FBI  \vspace{0.10cm}
\item After: arrest $\rightarrow$ FPs to FBI and ICE $\rightarrow$ if illegal, ICE takes custody \& deports\vspace{0.10cm}
\item  400,000 removals/year in 2014 - vast majority Latina/os
\end{itemize}\vspace{0.20cm}
\item Very unpopular and criticized, discontinued in 2012\vspace{0.20cm}
\item Replaced by Priority Enforcement Program (limited scope)
\end{itemize}
\end{frame}




%Literature:
%correlational studies between immigration and crime:  Bitcher and Piehl, 2007, Reid et al,. 2005, Graif and Sampson 2009. More causal: Bell Fasani Machin 2013, Chalfin (uses rainfall in Mexico). Most of them find zero. Also immigrants substantially less likely to be incarcerated than natives.
%Immigration status and crime: Pinotti and Mastrobuoni 2013, Pinotti 2017, Baker 2014, Freedman Owens and Bohn 2014.
%Miles and Cox focus on effects of DEPORTATION rates on crime - find zero. The secure communities didn't really affect perception of people because it was mandatory and did not involve action by the LEA (it was automatically implemented through the system of transmission of fingerprints to FBI and DHS). Sanctuary cities are more salient to people and are about refusing to cooperate

\begin{frame}{Tools for cooperation between LEA and ICE}
\begin{itemize}
\item \textbf{\textcolor{Blue}{Detainers}} - LEA keeps persons in custody after release to facilitate ICE's intervention\vspace{0.10cm}
\item \textbf{\textcolor{Blue}{Alerts}} - LEA notifies ICE when particular persons will be released\vspace{0.10cm}
\item \textbf{\textcolor{Blue}{287(g) agreements}} - training for local officials to enforce immigration law\vspace{0.10cm}
\item ICE in \textbf{\textcolor{Blue}{local jails}} without warrants\vspace{0.10cm}
\item Local officials can \textbf{\textcolor{Blue}{inquire about immigration status}}\vspace{0.10cm}
\item (in general) Use of local \textbf{\textcolor{Blue}{resources}} in assisting with immigration enforcement\vspace{0.10cm}
\end{itemize}
\end{frame}

\begin{frame}{Anti-cooperation policies}
\begin{center}
\includegraphics[width = \textwidth]{map_total_upwork.png}
\\\tiny{Source: Immigrant Legal Resource Center, 2016.}
\end{center}
\end{frame}

\begin{frame}{Cooperation between LEA and ICE}
\begin{itemize}
\item \textbf{Detainers - LEA keeps persons in custody after release to facilitate ICE's intervention}\vspace{0.10cm}
\item Alerts - LEA notifies ICE when particular persons will be released\vspace{0.10cm}
\item \textbf{287(g) agreements - training for local officials to enforce immigration law (future)}\vspace{0.10cm}
\item ICE in local jails without warrants\vspace{0.10cm}
\item Local officials can inquire about immigration status\vspace{0.10cm}
\item (in general)  Use of local resources in assisting with immigration enforcement
\end{itemize}
\end{frame}

\section{Data}
\begin{frame}{Data}
\begin{itemize}
\item Crime statistics: UCR, county-month, 2000-2016\vspace{0.10cm}\begin{itemize}
\item Counts of cleared, uncleared, unfound offenses by type\vspace{0.10cm}
\item Arrests
\end{itemize}\vspace{0.20cm}
\item Immigration/cooperation policies\vspace{0.10cm}
\begin{itemize}
\item No-detainer policies: ICE list with dates\vspace{0.10cm}
\item Secure Communities - enactment dates\vspace{0.10cm}
\item (future) 287(g) agreements - FOIA request
\end{itemize}\vspace{0.20cm}
\item Controls: population, migration, unemployment
\begin{itemize}
\item Population: 2000 and 2010 Census \vspace{0.10cm}
\item Immigration: Statistics of Income \vspace{0.10cm}
\item Unemployment / labor force: BLS
\end{itemize}
\end{itemize}
\end{frame}

\section{Research design}
\begin{frame}{Research design}
\begin{itemize}
\item Challenge: policies are not random - LEA's decision
\item Need to find credible control group
\begin{enumerate}
\item Bordering counties w/out no-detainer policy
\item Synthetic controls
\end{enumerate}
\end{itemize}
\end{frame}

\begin{frame}{No-detainer counties and bordering counties}
\begin{center}
\includegraphics[width = \textwidth]{map_detainer.png}
\end{center}
\end{frame}

\begin{frame}{No-detainer counties: Characteristics}
\begin{center}
Table
\end{center}
\end{frame}

\begin{frame}{Hispanic Population}
\begin{center}
\includegraphics[width = 0.8\textwidth]{hisp_slides1}
\end{center}
\end{frame}

\begin{frame}{Hispanic Population}
\begin{center}
\includegraphics[width = 0.8\textwidth]{hisp_slides2}
\end{center}
\end{frame}

\begin{frame}{Hispanic Population}
\begin{center}
\includegraphics[width = 0.8\textwidth]{hisp_slides3}
\end{center}
\end{frame}

\begin{frame}{Income}
\begin{center}
\includegraphics[width = 0.8\textwidth]{income_slides1}
\end{center}
\end{frame}

\begin{frame}{Income}
\begin{center}
\includegraphics[width = 0.8\textwidth]{income_slides2}
\end{center}
\end{frame}

\begin{frame}{Income}
\begin{center}
\includegraphics[width = 0.8\textwidth]{income_slides3}
\end{center}
\end{frame}

\begin{frame}{Democratic votes}
\begin{center}
\includegraphics[width = 0.8\textwidth]{dem_slides1}
\end{center}
\end{frame}

\begin{frame}{Democratic votes}
\begin{center}
\includegraphics[width = 0.8\textwidth]{dem_slides2}
\end{center}
\end{frame}

\begin{frame}{Democratic votes}
\begin{center}
\includegraphics[width = 0.8\textwidth]{dem_slides3}
\end{center}
\end{frame}

\begin{frame}{Immigration rates}
\begin{center}
\includegraphics[width = 0.8\textwidth]{migr_slides1}
\end{center}
\end{frame}

\begin{frame}{Immigration rates}
\begin{center}
\includegraphics[width = 0.8\textwidth]{migr_slides2}
\end{center}
\end{frame}

\begin{frame}{Immigration rates}
\begin{center}
\includegraphics[width = 0.8\textwidth]{migr_slides3}
\end{center}
\end{frame}

\begin{frame}{Crime}
\begin{center}
\includegraphics[width = 0.8\textwidth]{crime_slides1}
\end{center}
\end{frame}

\begin{frame}{Crime}
\begin{center}
\includegraphics[width = 0.8\textwidth]{crime_slides2}
\end{center}
\end{frame}

\begin{frame}{Crime}
\begin{center}
\includegraphics[width = 0.8\textwidth]{crime_slides3}
\end{center}
\end{frame}

\section{Effects of no-detainer on crime}
\begin{frame}{Trends in crime rates}
\begin{center}
\includegraphics[width = 0.8\textwidth]{crime_rate_levels_eventstudy}\\
\footnotesize{Note: crime rates per 100,000, conditional on no-ICE, quarter-year}
\end{center}
\end{frame}

\begin{frame}{Trends in crime rates - controlling for pre-trends}
\begin{center}
\includegraphics[width = 0.8\textwidth]{crime_rate_levels_eventstudy_nopretrends}\\
\footnotesize{Note: crime rates per 100,000, conditional on no-ICE, quarter-year}
\end{center}
\end{frame}

\begin{frame}{Total crime}
\footnotesize
\begin{center}
OLS - Dependent variable is ln(offenses/100,000)\\
{
\def\sym#1{\ifmmode^{#1}\else\(^{#1}\)\fi}
\begin{tabular*}{0.7\textwidth}{@{\hskip\tabcolsep\extracolsep\fill}l*{4}{c}}
\hline\hline
                    &\multicolumn{1}{c}{(1)}         &\multicolumn{1}{c}{(2)}         &\multicolumn{1}{c}{(3)}         &\multicolumn{1}{c}{(4)}         \\
\hline
no-detainer $\times$ post&      0.0286         &      0.0238\sym{*}  &      0.0222         &      0.0146         \\
                    &    (0.0201)         &    (0.0142)         &    (0.0226)         &    (0.0147)         \\
[1em]
County FE           &         Yes         &         Yes         &         Yes         &         Yes         \\
[1em]
Time FE             &          No         &         Yes         &          No         &         Yes         \\
[1em]
Controls            &          No         &         Yes         &          No         &         Yes         \\
[1em]
Trends              &          No         &          No         &         Yes         &         Yes         \\
\hline
Observations        &        6620         &        6569         &        6620         &        6569         \\
\hline\hline
\end{tabular*}
}

\\\footnotesize{Note: SEs clustered at the county level}
\end{center}
\end{frame}

\begin{frame}{Total crime}
\footnotesize
\begin{center}
OLS - Dependent variable is ln(offenses/100,000)\\
{
\def\sym#1{\ifmmode^{#1}\else\(^{#1}\)\fi}
\begin{tabular*}{0.7\textwidth}{@{\hskip\tabcolsep\extracolsep\fill}l*{4}{c}}
\hline\hline
                    &\multicolumn{1}{c}{(1)}         &\multicolumn{1}{c}{(2)}         &\multicolumn{1}{c}{(3)}         &\multicolumn{1}{c}{(4)}         \\
\hline
no-detainer $\times$ post&      0.0148         &      0.0197\sym{**} &      0.0134         &      0.0148         \\
                    &         (.)         &    (0.0094)         &    (0.0102)         &    (0.0155)         \\
[1em]
County FE           &         Yes         &         Yes         &         Yes         &         Yes         \\
[1em]
Group-by-year FE    &         Yes         &         Yes         &         Yes         &         Yes         \\
[1em]
Time FE             &          No         &         Yes         &          No         &         Yes         \\
[1em]
Controls            &          No         &         Yes         &          No         &         Yes         \\
[1em]
Trends              &          No         &          No         &         Yes         &         Yes         \\
\hline
Observations        &        6620         &        6569         &        6620         &        6569         \\
\hline\hline
\end{tabular*}
}

\\\footnotesize{Note: SEs clustered at the county level}
\end{center}
\end{frame}

\begin{frame}{Violent crime}
\footnotesize
\begin{center}
OLS - Dependent variable is ln(violent offenses/100,000)\\
{
\def\sym#1{\ifmmode^{#1}\else\(^{#1}\)\fi}
\begin{tabular}{l*{4}{c}}
\hline\hline
                    &\multicolumn{1}{c}{(1)}         &\multicolumn{1}{c}{(2)}         &\multicolumn{1}{c}{(3)}         &\multicolumn{1}{c}{(4)}         \\
\hline
no-detainer $\times$ post&      0.0054         &      0.0074         &      0.0049         &      0.0088         \\
                    &    (0.0267)         &    (0.0181)         &    (0.0260)         &    (0.0175)         \\
[1em]
County FE           &         Yes         &         Yes         &         Yes         &         Yes         \\
[1em]
Time FE             &          No         &         Yes         &          No         &         Yes         \\
[1em]
Controls            &          No         &         Yes         &          No         &         Yes         \\
[1em]
Trends              &          No         &          No         &         Yes         &         Yes         \\
\hline
Observations        &        6103         &        6062         &        6103         &        6062         \\
\hline\hline
\end{tabular}
}

\\\footnotesize{Note: SEs clustered at the county level}
\end{center}
\end{frame}

\begin{frame}{Violent crime}
\footnotesize
\begin{center}
OLS - Dependent variable is ln(violent offenses/100,000)\\
{
\def\sym#1{\ifmmode^{#1}\else\(^{#1}\)\fi}
\begin{tabular}{l*{4}{c}}
\hline\hline
                    &\multicolumn{1}{c}{(1)}         &\multicolumn{1}{c}{(2)}         &\multicolumn{1}{c}{(3)}         &\multicolumn{1}{c}{(4)}         \\
\hline
no-detainer $\times$ post&      0.0015         &      0.0176\sym{**} &     -0.0046         &      0.0164\sym{**} \\
                    &    (0.0167)         &    (0.0087)         &    (0.0183)         &    (0.0079)         \\
[1em]
County FE           &         Yes         &         Yes         &         Yes         &         Yes         \\
[1em]
Group-by-year FE    &         Yes         &         Yes         &         Yes         &         Yes         \\
[1em]
Time FE             &          No         &         Yes         &          No         &         Yes         \\
[1em]
Controls            &          No         &         Yes         &          No         &         Yes         \\
[1em]
Trends              &          No         &          No         &         Yes         &         Yes         \\
\hline
Observations        &        6103         &        6062         &        6103         &        6062         \\
\hline\hline
\end{tabular}
}

\\\footnotesize{Note: SEs clustered at the county level}
\end{center}
\end{frame}

\begin{frame}{Property crime}
\footnotesize
\begin{center}
OLS - Dependent variable is ln(property offenses/100,000)\\
{
\def\sym#1{\ifmmode^{#1}\else\(^{#1}\)\fi}
\begin{tabular*}{0.7\textwidth}{@{\hskip\tabcolsep\extracolsep\fill}l*{4}{c}}
\hline\hline
                    &\multicolumn{1}{c}{(1)}         &\multicolumn{1}{c}{(2)}         &\multicolumn{1}{c}{(3)}         &\multicolumn{1}{c}{(4)}         \\
\hline
no-detainer $\times$ post&      0.0322         &      0.0281\sym{*}  &      0.0204         &      0.0120         \\
                    &    (0.0212)         &    (0.0166)         &    (0.0235)         &    (0.0160)         \\
[1em]
County FE           &         Yes         &         Yes         &         Yes         &         Yes         \\
[1em]
Time FE             &          No         &         Yes         &          No         &         Yes         \\
[1em]
Controls            &          No         &         Yes         &          No         &         Yes         \\
[1em]
Trends              &          No         &          No         &         Yes         &         Yes         \\
\hline
Observations        &        6609         &        6564         &        6609         &        6564         \\
\hline\hline
\end{tabular*}
}

\\\footnotesize{Note: SEs clustered at the county level}
\end{center}
\end{frame}

\begin{frame}{Property crime}
\footnotesize
\begin{center}
OLS - Dependent variable is ln(property offenses/100,000)\\
{
\def\sym#1{\ifmmode^{#1}\else\(^{#1}\)\fi}
\begin{tabular}{l*{4}{c}}
\hline\hline
                    &\multicolumn{1}{c}{(1)}         &\multicolumn{1}{c}{(2)}         &\multicolumn{1}{c}{(3)}         &\multicolumn{1}{c}{(4)}         \\
\hline
no-detainer $\times$ post&      0.0328\sym{**} &      0.0418\sym{**} &      0.0259\sym{*}  &      0.0375\sym{**} \\
                    &    (0.0163)         &    (0.0169)         &    (0.0150)         &    (0.0151)         \\
[1em]
County FE           &         Yes         &         Yes         &         Yes         &         Yes         \\
[1em]
Group-by-time FE    &         Yes         &         Yes         &         Yes         &         Yes         \\
[1em]
Time FE             &         Yes         &         Yes         &         Yes         &         Yes         \\
[1em]
Controls            &          No         &         Yes         &          No         &         Yes         \\
[1em]
Trends              &          No         &          No         &         Yes         &         Yes         \\
\hline
Observations        &        6157         &        6112         &        6157         &        6112         \\
\hline\hline
\end{tabular}
}

\\\footnotesize{Note: SEs clustered at the county level}
\end{center}
\end{frame}
%
%
\begin{frame}{Total crime - by Share Hispanic}
\footnotesize
\begin{center}
OLS coefficients of \textit{no detainer $\times$ post $\times$ qtile hispanic share}\\
\includegraphics[width = 0.8\textwidth]{logcrime_dd_byhisp1.png}\\
\footnotesize{Note: SEs clustered at the county level}
\end{center}
\end{frame}

\begin{frame}{Total crime - by Share Hispanic}
\footnotesize
\begin{center}
OLS coefficients of \textit{no detainer $\times$ post $\times$ qtile hispanic share}\\
\includegraphics[width = 0.8\textwidth]{logcrime_dd_byhisp2.png}\\
\footnotesize{Note: SEs clustered at the county level}
\end{center}
\end{frame}

\begin{frame}{Total crime - by Share Hispanic}
\footnotesize
\begin{center}
OLS coefficients of \textit{no detainer $\times$ post $\times$ qtile hispanic share}\\
\includegraphics[width = 0.8\textwidth]{logcrime_dd_byhisp3.png}\\
\footnotesize{Note: SEs clustered at the county level}
\end{center}
\end{frame}

\begin{frame}{Violent crime - by Share Hispanic}
\footnotesize
\begin{center}
OLS coefficients of \textit{no detainer $\times$ post $\times$ qtile hispanic share}\\
\includegraphics[width = 0.8\textwidth]{logviolent_dd_byhisp1.png}\\
\footnotesize{Note: SEs clustered at the county level}
\end{center}
\end{frame}

\begin{frame}{Violent crime - by Share Hispanic}
\footnotesize
\begin{center}
OLS coefficients of \textit{no detainer $\times$ post $\times$ qtile hispanic share}\\
\includegraphics[width = 0.8\textwidth]{logviolent_dd_byhisp2.png}\\
\footnotesize{Note: SEs clustered at the county level}
\end{center}
\end{frame}

\begin{frame}{Violent crime - by Share Hispanic}
\footnotesize
\begin{center}
OLS coefficients of \textit{no detainer $\times$ post $\times$ qtile hispanic share}\\
\includegraphics[width = 0.8\textwidth]{logviolent_dd_byhisp3.png}\\
\footnotesize{Note: SEs clustered at the county level}
\end{center}
\end{frame}

\begin{frame}{Property crime - by Share Hispanic}
\footnotesize
\begin{center}
OLS coefficients of \textit{no detainer $\times$ post $\times$ qtile hispanic share}\\
\includegraphics[width = 0.8\textwidth]{logproperty_dd_byhisp1.png}\\
\footnotesize{Note: SEs clustered at the county level}
\end{center}
\end{frame}

\begin{frame}{Property crime - by Share Hispanic}
\footnotesize
\begin{center}
OLS coefficients of \textit{no detainer $\times$ post $\times$ qtile hispanic share}\\
\includegraphics[width = 0.8\textwidth]{logproperty_dd_byhisp2.png}\\
\footnotesize{Note: SEs clustered at the county level}
\end{center}
\end{frame}

\begin{frame}{Property crime - by Share Hispanic}
\footnotesize
\begin{center}
OLS coefficients of \textit{no detainer $\times$ post $\times$ qtile hispanic share}\\
\includegraphics[width = 0.8\textwidth]{logproperty_dd_byhisp3.png}\\
\footnotesize{Note: SEs clustered at the county level}
\end{center}
\end{frame}


% things do to: include time-varying controls, county-specific time trends (Miles and Cox show that they matter!!), and control for secure communities, see if the effect is different for counties with high share of immigrants






\end{document}