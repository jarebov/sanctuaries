\documentclass[xcolor=pdftex,dvipsnames,table]{beamer}
%\documentclass[xcolor=pdftex,dvipsnames,table,handout]{beamer}
% EEHPT: You can change the theme color from Green to other colors.
\usecolortheme[named=Blue]{structure}
\useinnertheme{circles}
\setbeamertemplate{enumerate items}[default]
\setbeamertemplate{section in toc}[default]
%\usetheme{Default}
%\usepackage{times}
%\setbeamerfont{title}{shape=\itshape,family=\rmfamily}
\setbeamercovered{invisible}
\usepackage{wrapfig}
\usepackage[absolute,overlay]{textpos}
\usepackage{array}
\usepackage{threeparttablex}
\newcolumntype{C}[1]{>{\centering\let\newline\\\arraybackslash\hspace{0pt}}m{#1}}
\newcolumntype{L}[1]{>{\raggedright\let\newline\\\arraybackslash\hspace{0pt}}m{#1}}
%% THE FOLLOWING TO HAVE TIMES NEW ROMAN
%\setbeamertemplate{frametitle}{
%\begin{centering} 
%\rmfamily\scshape \insertframetitle\par\vskip-6pt%\hrulefill
%\end{centering}} 
%\usefonttheme{serif}

\setbeamertemplate{frametitle}{
\begin{centering} 
\insertframetitle\par%\hrulefill
\end{centering}}


\usepackage {tikz}
\usepackage{textcomp}
\usetikzlibrary{positioning}
%\usepackage {xcolor}
\usetikzlibrary{shapes,arrows}

\usepackage{multirow}
%\usepackage[T1]{fontenc}
%\usepackage{txfonts}
\newcommand\Fontv{\fontsize{5}{5.5}\selectfont}
\newcommand\Fontvi{\fontsize{6}{7.4}\selectfont}
\newcommand\Fontvii{\fontsize{7}{8.4}\selectfont}
\newcommand\Fontviii{\fontsize{8}{9.6}\selectfont}
\newcommand*\itemgray{%
  \item[\color{gray}\scalebox{0.9}{\textbullet}]}
  
  
  
  %%% THE LINE BELOW MUST BE CHANGED WITH FOLDER IN USE!!
\newcommand{\tablesfolder}{/Users/barbarabiasi/Dropbox/Research/sanctuaries/out}  


  %%% THE LINE BELOW MUST BE CHANGED WITH FOLDER IN USE!!
\graphicspath{{/Users/barbarabiasi/Dropbox/Research/sanctuaries/out/}}

%\usepackage{lmodern}
%\usepackage[beamer,customcolors]{hf-tikz}
%\usepackage{enumitem}
%\usepackage{xcolor}
%\newif\ifgooditem
%\gooditemtrue
%\newcommand\gooditem{\gooditemtrue\item}
%\newcommand\baditem{\gooditemfalse\item}

%\tikzset{hl/.style={
%    set fill color=white,
%    set border color=red!80!black,
%  },
%}  
%\setbeamertemplate{footline}[default]
\setbeamercolor{button}{bg=white,fg=Blue}
\setbeamertemplate{button}[circle]

\usepackage{tikz}
\usepackage{bbding}
%\usepackage{pifont}
\usepackage{wasysym}
\usepackage{amssymb}
\usepackage{tikz}
\usepackage{booktabs}
\usepackage{multirow}
\setbeamertemplate
 {footline}{\nonumber}
  
\AtBeginSection[]
{
   \begin{frame}
        \frametitle{Outline}
        \tableofcontents[currentsection, 
    hideothersubsections]
   \end{frame}
}

\AtBeginSubsection[]
{
   \begin{frame}
        \frametitle{Outline}
        \tableofcontents[currentsection,currentsubsection]
   \end{frame}
}

\setbeamercolor{section in toc}{fg=Blue}
\setbeamercolor{subsection in toc}{fg=Blue}
\setbeamertemplate{section in toc}[sections numbered]
\setbeamerfont{subsection in toc}{size=\small}




\author[Short Name (U ABC)]{%
  \texorpdfstring{%
    \begin{columns}
      \column{.3333\linewidth}
      \centering
      Jaime Arellano-Bover \\ Stanford
      \column{.3333\linewidth}
      \centering
      Barbara Biasi \\ Princeton
    \end{columns}
 }
 {Jaime Arellano-Bover, Barbara Biasi}
}


\beamertemplatenavigationsymbolsempty
\begin{document}

\title[Sanctuary cities and crime]{Sanctuary Cities and Crime}

\date{December 2017}

\maketitle

\begin{frame}{Sanctuary Cities}
\begin{itemize}
\item Policies and practices of local authorities and law enforcement agencies (LEA) to restrict cooperation w/ federal authorities (FA) to enforce immigration law
\begin{itemize}
\item Prohibition of asking about immigration status
\item Prohibition on cooperation with ICE
\item Refusal to temporarily hold individuals or to notify ICE after release
\item Refusal to sign 287(g) agreements - allow local officials to enforce immigration law
\end{itemize}
\end{itemize}
\end{frame}

\begin{frame}{Debate over Sanctuaries}
\begin{itemize}
\item Points against
\begin{itemize}
\item They are illegal
\begin{itemize}
\item 8 US code 1373: no LEA can have policies restricting local officials from communicating or exchanging information with the FA
\item 8 US code 1373: no one may harbor illegal aliens or shield them from detection by FA
\end{itemize}
\item They harm public safety - make it easier to release criminals who should be deported
\end{itemize}
\item Points in favor:
\begin{itemize}
\item LEA shouldn't be forced to do FA's job
\item They provide safe heavens for war refugees
\item They reinforce trust between LEA and community, which improves policing activities
\end{itemize}
\end{itemize}
\end{frame}

\begin{frame}{ Do sanctuary cities affect crime?}
\begin{itemize}
\item In general: Does cooperation between LEA and FA on enforcement of immigration law affects crime?
\item Today: Focus on refusals of ICE detainers, introduced in 91 counties between XXXX and 2016
\item Future: look at 287(g) agreements
\end{itemize}
\end{frame}

\begin{frame}{Literature}
\end{frame}

\begin{frame}{Cooperation between LEA and ICE}
\begin{itemize}
\item Detainers - LEA keeps persons in custody after release to facilitate ICE's intervention
\item Alerts - LEA notifies ICE when particular persons will be released
\item 287(g) agreements - training for local officials to enforce immigration law
\item ICE in local jails without warrants
\item Local officials can inquire about immigration status
\item (in general) Use of local resources in assisting with immigration enforcement
\end{itemize}
\end{frame}

\begin{frame}{Anti-cooperation policies}
\begin{center}
\includegraphics[width = \textwidth]{map_total_upwork.png}
\\\tiny{Source: Immigrant Legal Resource Center, 2016.}
\end{center}
\end{frame}

\begin{frame}{Cooperation between LEA and ICE}
\begin{itemize}
\item \textbf{Detainers - LEA keeps persons in custody after release to facilitate ICE's intervention}
\item Alerts - LEA notifies ICE when particular persons will be released
\item \textbf{287(g) agreements - training for local officials to enforce immigration law}
\item ICE in local jails without warrants
\item Local officials can inquire about immigration status
\item (in general)  Use of local resources in assisting with immigration enforcement
\end{itemize}
\end{frame}


\begin{frame}{Research design: border comparison}
\begin{itemize}
\item Compare counties which introduced no-detainer policy with bordering counties
\item Exploit timing of introduction
\end{itemize}
\end{frame}

\begin{frame}{No ICE detainer counties}
\includegraphics[width = \textwidth]{map_detainer.png}
\end{frame}

\begin{frame}{Comparison}
\tiny
\input{\tablesfolder/sumstats.tex}
\end{frame}

\begin{frame}{Crime and Policies}
\includegraphics[width = \textwidth]{offenses_policies.png}\\
\tiny{Note: average of all counties at the year level}
\end{frame}

\begin{frame}{Total crime, by time to policy - treated counties}
\includegraphics[width = \textwidth]{totcrime_by_month_treated.png}\\
\tiny{Note: data at the county-month level}
\end{frame}

\begin{frame}{Violent and property crime, by time to policy - treated counties}
\includegraphics[width = \textwidth]{violent_property_by_month_treated.png}\\
\tiny{Note: data at the county-month level}
\end{frame}


\begin{frame}{Total crime, by time to policy - treated counties}
\includegraphics[width = \textwidth]{totcrime_by_year_treated.png}\\
\tiny{Note: data at the county-year level}
\end{frame}

\begin{frame}{Violent and property crime, by time to policy - treated counties}
\includegraphics[width = \textwidth]{violent_property_by_year_treated.png}\\
\tiny{Note: data at the county-year level}
\end{frame}

\begin{frame}{Event study, treated vs. bordering - total crime}
\includegraphics[width = \textwidth]{eventstudy_totcrime_qts.png}\\
\tiny{Note: data at the county-border-month level. No border FE. Observations weighted by population.}
\end{frame}

\begin{frame}{Event study, treated vs. bordering - violent crime}
\includegraphics[width = \textwidth]{eventstudy_violent_qts.png}\\
\tiny{Note: data at the county-border-month level. No border FE. Observations weighted by population.}
\end{frame}

\begin{frame}{Event study, treated vs. bordering - property crime}
\includegraphics[width = \textwidth]{eventstudy_property_qts.png}\\
\tiny{Note: data at the county-border-month level. No border FE. Observations weighted by population.}
\end{frame}

\begin{frame}{Event study, treated vs. bordering - total crime}
\includegraphics[width = \textwidth]{eventstudy_totcrime_qts_border.png}\\
\tiny{Note: data at the county-border-month level. With border FE. Observations weighted by population divided by nr. borders.}
\end{frame}

\begin{frame}{Event study, treated vs. bordering - violent crime}
\includegraphics[width = \textwidth]{eventstudy_violent_qts_border.png}\\
\tiny{Note: data at the county-border-month level. With border FE. Observations weighted by population divided by nr. borders.}
\end{frame}

\begin{frame}{Event study, treated vs. bordering - property crime}
\includegraphics[width = \textwidth]{eventstudy_property_qts_border.png}\\
\tiny{Note: data at the county-border-month level. With border FE. Observations weighted by population divided by nr. borders.}
\end{frame}


\begin{frame}{Event study, treated vs. bordering - total crime}
\includegraphics[width = \textwidth]{eventstudy_totcrime_qts_collapsed_border.png}\\
\tiny{Note: data at the county-border-QUARTER level. With border FE. Observations weighted by population divided by nr. borders.}
\end{frame}

\begin{frame}{Event study, treated vs. bordering - violent crime}
\includegraphics[width = \textwidth]{eventstudy_violent_qts_collapsed_border.png}\\
\tiny{Note: data at the county-border-QUARTER level. With border FE. Observations weighted by population divided by nr. borders.}
\end{frame}

\begin{frame}{Event study, treated vs. bordering - property crime}
\includegraphics[width = \textwidth]{eventstudy_property_qts_collapsed_border.png}\\
\tiny{Note: data at the county-border-QUARTER level. With border FE. Observations weighted by population divided by nr. borders.}
\end{frame}

\begin{frame}{Event study, treated vs. bordering - total crime}
\includegraphics[width = \textwidth]{eventstudy_totcrime_yr_collapsed_border.png}\\
\tiny{Note: data at the county-border-YEAR level. With border FE. Observations weighted by population divided by nr. borders.}
\end{frame}

\begin{frame}{Event study, treated vs. bordering - violent crime}
\includegraphics[width = \textwidth]{eventstudy_violent_yr_collapsed_border.png}\\
\tiny{Note: data at the county-border-YEAR level. With border FE. Observations weighted by population divided by nr. borders.}
\end{frame}

\begin{frame}{Event study, treated vs. bordering - property crime}
\includegraphics[width = \textwidth]{eventstudy_property_yr_collapsed_border.png}\\
\tiny{Note: data at the county-border-YEAR level. With border FE. Observations weighted by population divided by nr. borders.}
\end{frame}



\end{document}