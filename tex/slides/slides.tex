\documentclass[xcolor=pdftex,dvipsnames,table]{beamer}
%\documentclass[xcolor=pdftex,dvipsnames,table,handout]{beamer}
% EEHPT: You can change the theme color from Green to other colors.
\usecolortheme[named=Blue]{structure}
\useinnertheme{circles}
\setbeamertemplate{enumerate items}[default]
\setbeamertemplate{section in toc}[default]
%\usetheme{Default}
%\usepackage{times}
%\setbeamerfont{title}{shape=\itshape,family=\rmfamily}
\setbeamercovered{invisible}
\usepackage{wrapfig}
\usepackage[absolute,overlay]{textpos}
\usepackage{array}
\usepackage{threeparttablex}
\newcolumntype{C}[1]{>{\centering\let\newline\\\arraybackslash\hspace{0pt}}m{#1}}
\newcolumntype{L}[1]{>{\raggedright\let\newline\\\arraybackslash\hspace{0pt}}m{#1}}
%% THE FOLLOWING TO HAVE TIMES NEW ROMAN
%\setbeamertemplate{frametitle}{
%\begin{centering} 
%\rmfamily\scshape \insertframetitle\par\vskip-6pt%\hrulefill
%\end{centering}} 
%\usefonttheme{serif}

\setbeamertemplate{frametitle}{
\begin{centering} 
\insertframetitle\par%\hrulefill
\end{centering}}


\usepackage {tikz}
\usepackage{textcomp}
\usetikzlibrary{positioning}
%\usepackage {xcolor}
\usetikzlibrary{shapes,arrows}

\usepackage{multirow}
%\usepackage[T1]{fontenc}
%\usepackage{txfonts}
\newcommand\Fontv{\fontsize{5}{5.5}\selectfont}
\newcommand\Fontvi{\fontsize{6}{7.4}\selectfont}
\newcommand\Fontvii{\fontsize{7}{8.4}\selectfont}
\newcommand\Fontviii{\fontsize{8}{9.6}\selectfont}
\newcommand*\itemgray{%
  \item[\color{gray}\scalebox{0.9}{\textbullet}]}
  
  
  
  %%% THE LINE BELOW MUST BE CHANGED WITH FOLDER IN USE!!
\newcommand{\tablesfolder}{/Users/bbiasi/Dropbox/Research/sanctuaries/out}  


  %%% THE LINE BELOW MUST BE CHANGED WITH FOLDER IN USE!!
\graphicspath{{/Users/barbarabiasi/bbiasi/Research/sanctuaries/out/}}

%\usepackage{lmodern}
%\usepackage[beamer,customcolors]{hf-tikz}
%\usepackage{enumitem}
%\usepackage{xcolor}
%\newif\ifgooditem
%\gooditemtrue
%\newcommand\gooditem{\gooditemtrue\item}
%\newcommand\baditem{\gooditemfalse\item}

%\tikzset{hl/.style={
%    set fill color=white,
%    set border color=red!80!black,
%  },
%}  
%\setbeamertemplate{footline}[default]
\setbeamercolor{button}{bg=white,fg=Blue}
\setbeamertemplate{button}[circle]

\usepackage{tikz}
\usepackage{bbding}
%\usepackage{pifont}
\usepackage{wasysym}
\usepackage{amssymb}
\usepackage{tikz}
\usepackage{booktabs}
\usepackage{multirow}
\setbeamertemplate
 {footline}{\nonumber}
  
\AtBeginSection[]
{
   \begin{frame}
        \frametitle{Outline}
        \tableofcontents[currentsection, 
    hideothersubsections]
   \end{frame}
}

\AtBeginSubsection[]
{
   \begin{frame}
        \frametitle{Outline}
        \tableofcontents[currentsection,currentsubsection]
   \end{frame}
}

\setbeamercolor{section in toc}{fg=Blue}
\setbeamercolor{subsection in toc}{fg=Blue}
\setbeamertemplate{section in toc}[sections numbered]
\setbeamerfont{subsection in toc}{size=\small}




\author[Short Name (U ABC)]{%
  \texorpdfstring{%
    \begin{columns}
      \column{.3333\linewidth}
      \centering
      Jaime Arellano-Bover \\ Stanford
      \column{.3333\linewidth}
      \centering
      Barbara Biasi \\ Princeton
    \end{columns}
 }
 {Jaime Arellano-Bover, Barbara Biasi}
}


\beamertemplatenavigationsymbolsempty
\begin{document}

\title[Sanctuary cities and crime]{Sanctuary Cities and Crime}

\date{December 2017}

\maketitle

\begin{frame}{Sanctuary Cities}
\begin{itemize}
\item Policies and practices of local authorities and law enforcement agencies (LEA) to restrict cooperation w/ federal authorities (FA) to enforce immigration law
\begin{itemize}
\item Prohibition of asking about immigration status
\item Prohibition on cooperation with ICE
\item Refusal to temporarily hold individuals or to notify ICE after release
\item Refusal to sign 287(g) agreements - allow local officials to enforce immigration law
\end{itemize}
\end{itemize}
\end{frame}

\begin{frame}{Debate over Sanctuaries}
\begin{itemize}
\item Points against
\begin{itemize}
\item They are illegal
\begin{itemize}
\item 8 US code 1373: no LEA can have policies restricting local officials from communicating or exchanging information with the FA
\item 8 US code 1373: no one may harbor illegal aliens or shield them from detection by FA
\end{itemize}
\item They harm public safety - make it easier to release criminals who should be deported
\end{itemize}
\item Points in favor:
\begin{itemize}
\item LEA shouldn't be forced to do FA's job
\item They provide safe heavens for war refugees
\item They reinforce trust between LEA and community, which improves policing activities
\end{itemize}
\end{itemize}
\end{frame}

\begin{frame}{ Do sanctuary cities affect crime?}
\begin{itemize}
\item In general: Does cooperation between LEA and FA on enforcement of immigration law affects crime?
\item Today: Focus on refusals of ICE detainers, introduced in XX counties between XXXX and 2016
\item Future: look at 287(g) agreements
\end{itemize}
\end{frame}

\begin{frame}{Literature}
\begin{itemize}
\item Immigration and crime: Butcher and Piel (2007), Baker (2015), Pinotti (2017)
\item LEA/FA cooperation and crime: Miles and Cox (2014, 2015)
% Note: Miles&Cox (2014) use SecureCommunities program to study direct effect on crime. The 2015 article checks whether police officers find crimes at a higher rate, to test for cooperation. Both papers have a zero result
\end{itemize}
\end{frame}

\section{Institutional background}

\begin{frame}{Historical context}
\begin{itemize}
\item State and local police primary enforcers of criminal laws in US
\item Historically, immigration law competence of FA
\item 1996: Immigration and Nationality Act, Section 287(g)
\begin{itemize}
\item FA train LEA officers to enforce immigration law
\item Widely used until 2008
\end{itemize}
\item 2008: Secure Communities
\begin{itemize}
\item Limit LEA discretion
\item LEAs required to share info on arrested noncitizens
\item Removals peak at 400,000/year in 2014
\end{itemize}

\item LEA/FA cooperation and crime: Miles and Cox (2014, 2015)
% Note: Miles&Cox (2014) use SecureCommunities program to study direct effect on crime. The 2015 article checks whether police officers find crimes at a higher rate, to test for cooperation. Both papers have a zero result
\end{itemize}
\end{frame}

recent years have seen a dramatic increase in
direct state and local involvement in the enforcement of the federal immigration laws.5

 1996 immigration reforms creating
a new program under Immigration and Nationality Act Section 287(g)

Pursuant to so-called
Section 287(g) agreements entered into by federal, state, and local governments, federal
immigration authorities train state and local police in immigration enforcement and state and
local law enforcement are authorized to affirmatively assist federal immigration authorities in
enforcing the immigration laws. (widely used by Bush)

Moving away from the use of Section 287(g) agreements, the Obama administration
opted instead for the ?Secure Communities? program, which was designed to limit state and
local police discretion in immigration enforcement while still relying on the state and local
criminal justice systems to facilitate the removal of criminal noncitizens. required state and local 
5law enforcement agencies to cooperate with federal immigration enforcement authorities.57 State
and local police agencies shared information about noncitizens arrested with the federal
government and were instructed to place immigration ?holds? on (i.e., detain) noncitizens who
were arrested so that federal authorities had the time necessary to take custody of the noncitizens
for possible removal. was so effective that removals soared to record highs in the neighborhood of approximately
400,000 noncitizens a year in the first six years of the Obama presidency, regularly setting
records for the number of removals.
60 Secure Communities also resulted in the vast majority of
the persons removed being Latina/o immigrants

Concerned with the overbroad impacts and negative public safety implications of Secure
Communities, some states and localities began to resist cooperation with the federal government
in immigration enforcement and refused full participation in the program.
62 Such resistance
contributed to the Obama administration?s unceremonious dismantling of Secure Communities,
63
a step that was overshadowed in the public eye by the simultaneous announcement of a new
expanded deferred action program that provoked considerable controversy and legal 

?Priority Enforcement Program? (PEP) with the stated intent of
focusing removal efforts on serious criminal offenders; PEP also changed federal policy to
restrict requests for immigration ?holds? to noncitizens actually convicted of crimes rather than
merely arrested for them

\section{Institutional background}
\begin{frame}{Cooperation between LEA and ICE}
\begin{itemize}
\item Detainers - LEA keeps persons in custody after release to facilitate ICE's intervention
\item Alerts - LEA notifies ICE when particular persons will be released
\item 287(g) agreements - training for local officials to enforce immigration law
\item ICE in local jails without warrants
\item Local officials can inquire about immigration status
\item (in general) Use of local resources in assisting with immigration enforcement
\end{itemize}
\end{frame}

\begin{frame}{Cooperation between LEA and ICE}
\begin{itemize}
\item Detainers - LEA keeps persons in custody after release to facilitate ICE's intervention
\item Alerts - LEA notifies ICE when particular persons will be released
\item 287(g) agreements - training for local officials to enforce immigration law
\item ICE in local jails without warrants
\item Local officials can inquire about immigration status
\item (in general) Use of local resources in assisting with immigration enforcement
\end{itemize}
\end{frame}

\begin{frame}{Anti-cooperation policies}
\begin{center}
\includegraphics[width = \textwidth]{map_total_upwork.png}
\\\tiny{Source: Immigrant Legal Resource Center, 2016.}
\end{center}
\end{frame}

\begin{frame}{Cooperation between LEA and ICE}
\begin{itemize}
\item \textbf{Detainers - LEA keeps persons in custody after release to facilitate ICE's intervention}
\item Alerts - LEA notifies ICE when particular persons will be released
\item \textbf{287(g) agreements - training for local officials to enforce immigration law (future)}
\item ICE in local jails without warrants
\item Local officials can inquire about immigration status
\item (in general)  Use of local resources in assisting with immigration enforcement
\end{itemize}
\end{frame}

\section{Research design}

\begin{frame}{Research design}
\begin{itemize}
\item Challenge: policies are not random - LEA's decision
\item Need to find credible control group
\begin{enumerate}
\item Bordering counties w/out no-detainer policy
\item Synthetic controls
\end{enumerate}
\end{itemize}
\end{frame}

\begin{frame}{No-ICE counties and bordering counties}
\begin{center}
\includegraphics[width = \textwidth]{map_detainer.png}
\end{center}
\end{frame}



\end{document}