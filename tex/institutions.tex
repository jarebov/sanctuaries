\section{Institutional details}
``Sanctuary policies" is a popular term to define policies and practices enacted by local authorities, law enforcement agencies, and local governments to restrict or prohibit cooperation with federal immigration authorities in enforcing immigration law. These practices can take several forms, including impeding state and local officials, including law enforcement officers, from asking individuals about their immigration status, reporting them to the federal government, or otherwise cooperating with or assisting federal immigration officials. 

NOTE: While many of these policies and practices are written, they may be unwritten as well, sometimes making them difficult to discover or verify.

Most of the sanctuary policies and practices fall into the ?anti-detainer? category. These generally refer to directives that inhibit or restrict the ability of state and local law enforcement to hold criminal aliens for U.S. Immigration and Customs Enforcement (ICE). Some anti-detainer policies even go so far as to prohibit state and local law enforcement from simply notifying ICE that they are about to release a criminal alien back onto the streets or from otherwise assisting federal authorities.

Many deportable immigrants come to the attention of ICE through its various partnerships with local law enforcement agencies, including Secure Communities, the Criminal Alien Program, and 287(g). Through these programs, ICE targets individuals who have come into contact with local and state law enforcement. If ICE has reason to believe that an individual in criminal custody may be removable, it can issue an immigration detainer asking the local law enforcement agency to continue to hold that individual for up to 48 hours to give ICE a chance to place the person into immigration custody ? regardless of whether the person was ever convicted of a crime. Although immigration detainers are merely requests - not mandatory ? more often than not local law enforcement agents comply. The result is the deportation of increasing numbers of often innocent immigrants. According to recent data (http://trac.syr.edu/immigration/reports/330/), only 14\% of the detainers ICE issued in FY 2012 and during the first four months of FY 2013 ?target[ed] individuals who pose a serious threat to public safety or national security? while approximately half implicated individuals with ?no record of criminal conviction, not even a minor traffic violation.?

Note: from FAIR report, FAIR has found that jurisdictions often justify their sanctuary policies by claiming that illegal aliens will be more likely to report crimes to law enforcement without fear of deportation. However, FAIR knows of no evidence demonstrating that sanctuary policies lead to increased crime reporting among illegal immigrant communities, and law enforcement officers already have the discretion to grant immunity to witnesses and victims of crime. IT WOULD BE COOL TO SEE IF WE CAN FIND DATA ON CRIMES REPORTED!

Other policies:
Section 287(g) of the Immigration and Nationality Act (INA) ? has surged into public consciousness in recent years. Now operating in 72 jurisdictions, the 287(g) program authorizes state and local officers to screen people for immigration status, issue detainers to hold them on immigration violations until the federal government takes custody, and generate the charges that begin the process of their removal from the United States. In practice, 287(g) allows state and local agents to directly enforce federal immigration law. The 287(g) program includes three different models: (1) the jail model, in which officials screen
for immigration status and issue detainers when booking arrestees into jails on criminal (i.e. nonimmigration) charges; (2) the task force model, in which state and local officials screen for status and issue detainers in the field during policing operations; and (3) the hybrid model, in which jurisdictions maintain both jail and task force authority.

Section 287(g) authorizes the Attorney General (now the Secretary of Homeland Security, with the move of immigration enforcement from the Justice Department to the Department of Homeland Security)
to enter into written agreements with state and local officials authorizing the latter to ?perform the function of an immigration officer in relation to the investigation, apprehension, or detention of aliens in the United States (including the transportation of such aliens across State lines to detention centers)? at the expense of the state or locality.11 The statute makes clear that state and local officials must possess knowledge of federal law, receive training in federal enforcement, and be supervised and directed by federal officials. The statute also places state and local enforcement through the program ?under color of Federal authority for purposes of determining liability, and immunity from suit.?


In general, the program allows 287(g) officers to perform the following tasks:
- Check DHS databases for immigration status information.

- Interview immigrants to ascertain their status. 

- Enter data into ENFORCE, ICE?s database and case management system that collects and stores information on noncitizens encountered, detained, and removed through ICE?s various enforcement programs.

- Issue ICE detainers. An ICE detainer is a formal document that allows a law enforcement agency to hold the person for up to 48 hours until he or she can be transferred into ICE custody for removal processing. 

- Place immigration charges. 287(g) officers may use the ENFORCE case management system to issue a Notice to Appear (NTA), the official charging document that initiates formal removal proceedings and a hearing before an immigration judge to adjudicate the immigration charges. 

NOTE: the document says that 287(g) was started in Florida in 2002, then Alabama, Los Angeles county followed, then 4 more counties between 2006-2008.


Secure communities: Under Secure Communities, when arrestees are booked into state or local jails and their fingerprint data are sent to the FBI for criminal background checks, their fingerprints also are transmitted electronically to ICE?s Law Enforcement Support Center (LESC). ICE officers at the LESC check the fingerprints against data in several immigration databases and then notify ICE officers in local field offices if the person appears to be removable. Local ICE officers make decisions about whom to pursue on immigration charges.

Criminal Alien Program: Under the Criminal Alien Program, ICE officers are stationed in the state prison or local jail, where they conduct immigration screening. CAP officers also issue detainers and NTAs.

The Criminal Alien Program (CAP) is an umbrella program that includes systems for identifying
and initiating removal proceedings for priority criminal aliens who are incarcerated within
federal, state, and local prisons and jails, as well as at-large criminal aliens who have avoided identification. CAP is intended to prevent the release of criminal aliens from jails and prisons into
U.S. communities by securing final orders of removal either prior to the termination of aliens?
criminal sentences or subsequently whenever possible, and by taking custody of and removing
priority aliens who complete their criminal sentences.36 Identifying and processing incarcerated
criminal aliens before their release from jails and prisons is intended to reduce or eliminate time
spent in ICE custody and reduce related overall costs to the federal government.
CAP jail enforcement officers screen people to identify and prioritize potentially removable aliens
as they are being booked into jails and prisons and while they are serving their sentences. Such
screening covers almost all persons booked into federal and state prisons and local jails.37 CAP
officers search biometric and biographic databases to identify matches in DHS databases and
interview arrestees and prisoners to identify potentially removable aliens without DHS records.38
When CAP officers identify a removable alien, they may issue a request for notification to state
or local law enforcement agencies formally asking to be contacted prior to an alien?s release from
custody. Issuance of a request for notification depends on whether removal of the flagged
individual accords with CAP priorities.39 CAP officers may issue an immigration detainer if an
individual is subject to a final order of removal.
40
As of April 2016, approximately 1,300 CAP officers were monitoring 100\% of federal and state
prisons, a total of over 4,300 facilities.



